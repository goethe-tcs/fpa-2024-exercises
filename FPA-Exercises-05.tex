% LTeX: language=en_US
\documentclass[english]{uebung_cs}
\usepackage{settings}
\blattnummer{5}
\blattname{Problem set \theblattnummer: Treewidth and Graph minors}
%%%%%%%%%%%%%%%%%%%%%%%%%%%%%%%%%%%%%%%%%%%%%%%%%%%%%%%%%%%%%%%%%%%%%%%%%%%%
\begin{document}

\textbf{Overview:} With this problem set, you can train reasoning about algorithmic meta-theorems and graph minor theory.

\textbf{Instructions:} For each skill, select \textbf{exactly one} problem below and submit your solution in \href{https://moodle.studiumdigitale.uni-frankfurt.de/moodle/course/view.php?id=6259}{Moodle}; in your submission, make sure to repeat the problem that you are solving.
The problems are roughly ordered by difficulty, choose a problem that you find non-trivial and interesting. (You are of course welcome to try the other problems as well and ask us for feedback.)

% % own exercise, 2018
% \begin{exercise}[todo: skill]
% %[\easy]
% Let $n$ be a positive integer and consider the vector space $E_n$ which is defined to be the linear span (over $\mathbb{R}$) of the basis $\{e_S ~|~S \subseteq \{1,\dots,n\}\}$. That is, vectors of $E_n$ are of the form
% \[ \sum_{S \subseteq \{1,\dots,n\}} \lambda_S e_S \,,\]
% where $\lambda_S \in \mathbb{R}$ for every $S$. We define a vector multiplication $\wedge$ over (the basis of) $E_n$ as follows:
% \[e_S \wedge e_T = \varphi(S,T)\cdot e_{S \cup T} \,,\]
% where $\varphi(S,T)= 0$ if $S \cap T \neq \emptyset$ and $\varphi(S,T)=\mathsf{sgn}(\sigma_{S,T})$\footnote{The sign of a permutation is $(-1)^m$ where $m$ is the number of transpositions, e.g. $\mathsf{sgn}(14253)= \mathsf{sgn}((14)(42)(25)(53)) = (-1)^4 = 1$. } otherwise. Here $\sigma_{S,T}$ is the permutation that, given the sequence of elements of $S$ and $T$, each ordered, outputs the ordered sequence over all elements of $S\cup T$. \textbf{Example:} $\varphi(\{4,2\},\{2,3,1\})= 0$ and $\varphi(\{5,4\},\{2,3,1\}) = \mathsf{sgn}(14253)$ because the permutation $(14253)$ orders the sequence $4,5,1,2,3$, which again is the sequence of $\{5,4\}$ and $\{2,3,1\}$ where each set is ordered.

% Prove or disprove: For every $e \in E_n\setminus \{0\}$ it holds that $e^2 = 0$.
% \end{exercise}

\begin{skill}[Algorithmic meta-theorems][\mandatory]
  I can apply algorithmic meta-theorems to design fixed-parameter tractable algorithms.
\end{skill}

\begin{exercise}[Maximum Cut using Courcelle's Theorem]
  A \emph{cut} of a graph $G$ is a partition $(A,B)$ of the vertices of $G$ and the \emph{size} of a cut $(A,B)$ is the number of edges of $G$ that have one endpoint in $A$ and one endpoint in $B$. The problem $\mathrm{MaxCut}$ asks, given a graph $G$, to compute the size of the largest cut in $G$. Prove that $\mathrm{MaxCut}$ can be solved in time $f(\mathsf{tw}(G)) \cdot |V(G)|$ for some computable function $f$ by invoking Courcelle's Theorem (Theorem 7.11).
\end{exercise}

\begin{exercise}[Graph Property Verification using Courcelle's Theorem]
  %[\easy]
  Let $\Phi$ be a graph property expressible in monadic second-order logic. The problem $\textsc{Verify}[\Phi]$ asks, given a graph~$G$, to decide whether~$G$ has property~$\Phi$.
  Invoke Courcelle's Theorem (Theorem 7.11) to prove that this problem can be solved in time $f(\mathsf{vc}(G))\cdot n$ for some computable function $f$, where $\mathsf{vc}(G)$ is the size of the largest vertex-cover of~$G$.
\end{exercise}

% Cygan et al., Exercise 6.19
\begin{exercise}[Cycle Packing using Graph Minors]
  %[\easy]
  In the \emph{Cycle Packing} problem, we are given an undirected graph~$G$ and a positive integer $k$, and the goal is to check whether there exist $k$ cycles in $G$ that are pairwise vertex disjoint. Use Theorem~6.12 from the book to prove that this problem is nonuniformly fixed-parameter tractable when parameterized by $k$.
\end{exercise}

% Cygan et al., Exercise 6.10
\begin{exercise}[Closest String using Integer Linear Programming][\hard]
  In the \emph{Closest String} problem, we are given a set of $k$ strings $x_1,\dots,x_k$ over alphabet $\Sigma$, each of length $L$, and a positive integer $d$. The goal is to find a string $y$ of length $L$ such that the \emph{Hamming Distance}\footnote{The Hamming Distance between two strings $x$ and $y$ of the same length is the number of positions $i$ such that $x_i \neq y_i$ holds.} between $y$ and $x_i$ is bounded by $d$ for every $i\in \{1,\dots,k\}$. Use Theorem~6.5 from the book to prove that the problem is fixed-parameter tractable when parameterized by $k$ and $|\Sigma|$.

  \noindent\emph{Optional bonus task:} Show that this problem parameterized by~$k$ only remains fixed-parameter tractable.
\end{exercise}

\newpage
\begin{skill}[Graph minors][\mandatory]
  I can mathematically reason about graph minors and apply the Graph Minors Theorem. \normalfont (See Section~6.3 and~7.7 in \cygan{})
\end{skill}

% Cygan et al., Exercise 7.35
\begin{exercise}[Excluded Grid Theorem is Tight]
  Show that the dependency on $k$ in the Excluded Grid Theorem needs to be $\Omega(k^2)$. That is, construct a graph of treewidth $\Omega(k^2)$ that does not contain a $k \times k$ grid as a minor.
\end{exercise}

% Cygan et al., Exercise 7.40
\begin{exercise}[Bidimensionality]
  Show that the following problems are bidimensional: Feedback Vertex Set, Induced Matching, Cycle Packing, Scattered Set for a fixed value of $d$, Longest Path, Dominating Set, and $r$-Center for a fixed $r$.
\end{exercise}

% Cygan et al., Exercise 7.37
\begin{exercise}[Brambles in Grid Graphs][\hard]
  Prove that for every $t > 1$, the $t\times t$ grid graph contains a bramble of order $t + 1$ and thus the treewidth of the $t\times t$ grid graph is~$t$.
\end{exercise}

\end{document}
