% LTeX: language=en_US
\documentclass{uebung_cs}
\usepackage{settings}
\blattnummer{2}
\blattname{TODO}
%%%%%%%%%%%%%%%%%%%%%%%%%%%%%%%%%%%%%%%%%%%%%%%%%%%%%%%%%%%%%%%%%%%%%%%%%%%%
\begin{document}

\textbf{Note:} Problems marked with \mandatory are mandatory and must be submitted in \href{https://moodle.studiumdigitale.uni-frankfurt.de/moodle/course/view.php?id=6259}{Moodle}.

\begin{exercise}[Application of Sunflower Lemma: \itshape I can todo][\mandatory]
First, read Section 2.6 about the Sunflower Lemma in the book, including its application to the $d$-hitting set problem.
Then, using similar ideas, solve the following exercise on the $d$-set packing problem:
You are given an integer $k$ and a family of subsets $\mathcal{A}$ of a universe $U$, where each set in $\mathcal{A}$ is of size at most $d$. Decide whether there exist $k$ sets $S_1,\ldots,S_k \in \mathcal{A}$ that are pairwise disjoint.
Use the Sunflower Lemma to obtain a kernel for this problem with $f(d)\cdot k^d$ sets, for some computable $f$.
\end{exercise}

\begin{exercise}[todo: skill]
%[\easy]
Fix an integer $r > 0$. 
Show that the clique and independent set problem, when parameterized by the solution size, are fixed-parameter tractable on $r$-regular graphs.
Show that this remains true if $r$ is not fixed anymore and we parameterize by $k+r$ .
\end{exercise}

\begin{exercise}[todo: skill]
%[\easy]
Prove: The vertex cover problem can be solved optimally in polynomial time on graphs of maximum degree at most $2$.
\end{exercise}

\begin{exercise}[todo: skill]
Show that a graph on $n$ vertices of minimum degree at least $3$ contains a cycle of length at most $2\lceil \log n\rceil$.
Use this to design a $(\log n)^{O(k)} \cdot  n^{O(1)}$-time algorithm for the feedback vertex set problem on undirected graphs.
Does this runtime bound suffice the conditions for an fpt-algorithm, i.e., can it be bounded by $f(k) \cdot n^{O(1)}$ for some computable $f$?
\end{exercise}

\textbf{Note:} Problems marked with \hard are especially challenging---they require that you invest a lot of time, play around with different ideas, and have a bit of luck that you find one that works. Best enjoyed with your favorite beverage and a friend. If you solve them, you can be proud of yourself. If you don't, you can still be proud of yourself for trying.

\begin{exercise}[todo: skill][\hard]
A graph is \emph{chordal} if it does not contain a cycle on at least four vertices as an induced subgraph.
That is, for every cycle of length at least four, 
there is at least one edge in the graph between to vertices of the cycle that is not in the cycle. 
Such an edge is a \emph{chord}. Hence the name.

A \emph{triangulation} of a graph $G=(V,E)$ is a set of edges $E' \subseteq \binom{V}{2}$ such that $(V,E\cup E')$ is chordal.

Consider the chordal completion problem: Given a graph and an integer $k$, decide whether $G$ has a triangulation of size at most $k$. 
\begin{enumerate}
\item Give a $k^{O(k)} n^{O(1)}$-time algorithm for the problem.
\item Show that there is a bijection between inclusion-wise minimal triangulations of a cycle of length $\ell$ and binary trees with $\ell-2$ internal nodes. From this, conclude that a cycle on $\ell$ vertices has at most $4^{\ell-2}$ inclusion-wise minimal triangulations.
\item Use the previous point to design an algorithm for the problem running in time $2^{O(k)} \cdot n^{O(1)}$.
\end{enumerate}
\end{exercise}

\begin{exercise}[todo: skill][\hard]
Consider the following problem: Given a connected graph $G$ and an integer $k$, decide whether there is a spanning tree of $G$ with at least $k$ leaves. Design an algorithm that solves this problem in time $4^k \cdot n^{O(1)}$.
\end{exercise}

\newpage
\textbf{Note:} Problems marked with \schriftlich are meant to train your formal writing skills.

\begin{exercise}[Formal writing: \itshape I can write a formal proof when I am given a proof sketch][\schriftlich]
  todo: exercise text
\end{exercise}

\end{document}
