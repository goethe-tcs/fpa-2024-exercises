% LTeX: language=en_US
\documentclass[english]{uebung_cs}
\usepackage{settings}
\blattnummer{6}
\blattname{Problem set \theblattnummer: Algebraic Methods}
%%%%%%%%%%%%%%%%%%%%%%%%%%%%%%%%%%%%%%%%%%%%%%%%%%%%%%%%%%%%%%%%%%%%%%%%%%%%
\begin{document}

\textbf{Overview:} With this problem set, you can train reasoning about algebraic methods.

\textbf{Instructions:} For each skill, select \textbf{exactly one} problem below and submit your solution in \href{https://moodle.studiumdigitale.uni-frankfurt.de/moodle/course/view.php?id=6259}{Moodle}; in your submission, make sure to repeat the problem that you are solving.
The problems are roughly ordered by difficulty, choose a problem that you find non-trivial and interesting. (You are of course welcome to try the other problems as well and ask us for feedback.)

\begin{skill}[Reason about and adapt algebraic methods][\mandatory]
  I can formally reason about and adapt fast Möbius transforms and fast product operations. \normalfont (See Sections~10.1--10.3 in \cygan{})
\end{skill}

% Cygan et al., Exercise 10.3
\begin{exercise}[Proof of Proposition 10.10][easy]
  Prove that $\zeta = \sigma \mu \sigma$ and $\mu = \sigma \zeta \sigma$ hold.
\end{exercise}

% Cygan et al., Exercise 10.13
\begin{exercise}[Fast Packing Product]
  The \emph{packing product} of two functions \( f,g \colon 2^V \to \mathbb{Z} \) is a function \( (f \ast_p g) \colon 2^V \to \mathbb{Z} \) such that for every \( Y \subseteq V \), we have
  \begin{align*}
    (f \ast_p g)(Y) = \sum_{\substack{A,B \subseteq Y \\A\cap B =\emptyset}} f(A) g(B) \,.
  \end{align*}
  Show that all \( 2^{n} \) values of \( f \ast_p g \) can be computed in time \( 2^n n^{\Oh(1)} \), where \( n = |V| \).
\end{exercise}

\begin{exercise}[Möbius inversion on posets]%[\hard]
  \label{posets}
  In this guided exercise, we generalize the principle of Möbius inversion to finite partially ordered sets (posets). To this end, let $P$ be a poset. The \emph{incidence algebra} of a poset $(P,\leq)$ is defined as follows:
  \[\mathbb{I}(P,\leq) \coloneqq \{A \in \mathbb{C}^{P\times P} ~|~ x \nleq y \Rightarrow A(x,y)=0\} \,.\]
  One example of an element of $\mathbb{I}(P,\leq)$ is the so-called \emph{zeta function}:
  \[\zeta(x,y)=\begin{cases} 1 ~~~\text{if } x \leq y\,,\\0~~~\text{otherwise.} \end{cases} \]

  Consider the element $\mu$ of $\mathbb{I}(P,\leq)$, which is called the \emph{Möbius function} over $(P,\leq)$ and which is inductively defined as follows:
  \[\mu(x,y)=\begin{cases}
      1 ~~~~~~~~~~~~\text{if } x = y\,,\\0~~~~~~~~~~~~\text{if } x \nleq y\,,\\ -\sum_{x \leq z < y}\mu(x,z)~~~\text{otherwise.} \end{cases} \]
  \begin{itemize}
    \item Prove that the following identity holds for all $x,y \in P$:
          \[\sum_{x\leq z\leq y}\mu(x,z)=\begin{cases} 1 ~~~\text{if } x = y\\0~~~\text{otherwise} \end{cases} \]
    \item Prove $\mu = \zeta^{-1}$ and conclude from $\zeta\cdot \mu = \operatorname{id}$ that the following identity holds as well:
          \[\sum_{x\leq z\leq y}\mu(z,y)=\begin{cases} 1 ~~~\text{if } x = y\,,\\0~~~\text{otherwise.} \end{cases} \]
    \item \textbf{Möbius inversion:} Let $f,g \colon P \to \mathbb{C}$ such that $g(x) = \sum_{y\leq x} f(y)$ holds for all $x \in P$. Prove that the following identity holds for all $x \in P$:
          \[f(x) = \sum_{y \leq x} \mu(y,x) \cdot g(y) \]
  \end{itemize}
\end{exercise}


\begin{skill}[Apply algebraic methods to design algorithms][\mandatory]
  I can apply inclusion--exclusion, the fast Möbius transform, and fast product operations to design fast algorithms. \normalfont (See Sections~10.1--10.3 in \cygan{})
\end{skill}

% Cygan et al., Exercise 10.9
\begin{exercise}[Ryser's formula]
  Use the principle of inclusion--exclusion to design an algorithm which computes the number of perfect matchings in a given \(n\)-vertex bipartite graph in time \(2^{n/2} n^{\Oh(1)}\) and polynomial space.
\end{exercise}

% Cygan et al., Exercise 10.8
\begin{exercise}[List coloring] %[easy]
  In the List Coloring problem, we are given an $n$-vertex graph \( G \) and, for each vertex \( v \in V(G) \), there is a set (also called a list) of admissible colors \( L(v) \subseteq \{1,\dots,n\} \). The goal is to verify whether it is possible to find a proper vertex coloring \( c \colon V(G) \rightarrow \mathbb{N} \) of \( G \) such that for every vertex \( v \), we have \( c(v) \in L(v) \). In other words, \( L(v) \) is the set of colors allowed for \( v \).
  Show a \( 2^n n^{\Oh(1)} \)-time algorithm for List Coloring.
  \hint{Adapt the algorithm from Theorem 10.16.}
\end{exercise}

\begin{exercise}[Counting subgraphs]
  In this guided exercise, you will develop an efficient algorithm for computing the number of subgraphs of a given graph~\(G\) that are isomorphic to a fixed graph \( H \).

  Building on \ref{posets}, we use Möbius inversion on the partition lattice:
  Let $H$ be a graph with vertex set~$V$. Given two partitions $\sigma$ and $\rho$ of $V$, we write $\sigma \rightarrow \rho$ if $\rho$ can be obtained from $\sigma$ by joining two blocks of~$\sigma$. Example: For $\sigma = \{ \{1,4\}, \{2\}, \{3\} \}$, we have $\sigma \rightarrow \{ \{1,2,4\}, \{3\} \}$. Now let $\leq$ be the reflexive-transitive closure of $\rightarrow$, i.e., $\sigma \leq \rho$ if and only if there are $\sigma_1,\dots,\sigma_k$ with $\sigma \rightarrow \sigma_1 \rightarrow \dots \rightarrow \sigma_k \rightarrow\rho$. Note that $k$ might be zero.
  \begin{itemize}
    \item Let $P(V)$ be the set of partitions of $V$. Show that $(P(V),\leq)$ is a poset. This poset is called the \emph{partition lattice}. What is the minimum~$\bot$ and the maximum $\top$ of this poset?
    \item Given an element $\sigma \in P(V)$, the graph $H/\sigma$ is obtained from $H$ by contracting each block of $\sigma$ to a single vertex, deleting multiple edges, and keeping self-loops. Given a graph $G$, we let $\mathsf{Hom}(H,G)$ be the number of graph homomorphisms from $H$ to $G$ and let $\mathsf{Inj}(H,G)$ be the number of injective graph homomorphisms from $H$ to $G$. Use Möbius inversion to prove
          \begin{equation}\label{eq1}
            \mathsf{Inj}(H,G) = \sum_{\sigma \in P(V)} \mu(\bot,\sigma)\cdot \mathsf{Hom}(H/\sigma,G)\,.
          \end{equation}
    \item Given graphs $H$ and $G$, it is known that the number of subgraphs of $G$ that are isomorphic to~$H$ equals $\mathsf{Aut}^{-1}(H)\cdot \mathsf{Inj}(H,G)$, where $\mathsf{Aut}(H)$ is the number of automorphisms of~$H$, that is, bijective homomorphisms from~$H$ to~$H$. Combine this knowledge with Exercise~4.7 and \eqref{eq1} to design an algorithm for $\textsc{Subgraph Isomorphism}$. What is the running time of your algorithm?
  \end{itemize}
\end{exercise}

\end{document}
