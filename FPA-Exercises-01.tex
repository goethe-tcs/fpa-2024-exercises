% LTeX: language=en_US
\documentclass{uebung_cs}
\usepackage{settings}
\blattnummer{1}
\blattname{Problem Set~\theblattnummer: Kernelization}
%%%%%%%%%%%%%%%%%%%%%%%%%%%%%%%%%%%%%%%%%%%%%%%%%%%%%%%%%%%%%%%%%%%%%%%%%%%%
\begin{document}

\textbf{Note:} Problems marked with \mandatory are mandatory and must be submitted in \href{https://moodle.studiumdigitale.uni-frankfurt.de/moodle/course/view.php?id=6259}{Moodle}.

\begin{exercise}[Falsify kernelization: \itshape I can precisely explain why a proposed kernelization algorithm for a given problem is not correct][\mandatory]
  \label{ex:CVC}
  The problem Connected Vertex-Cover is defined as follows: Given an undirected graph~$G$ and a positive integer~$k$, decide whether there exists a vertex set $C\subseteq V(G)$ such that $C$ has size at most~$k$ vertices, the subgraph~$G[C]$ induced by $C$ is connected, and $C$~is a vertex-cover of~$G$.
  Where do the kernelization rules for Vertex-Cover (see Theorem 2.4) fail when applied to Connected Vertex-Cover?
\end{exercise}

\begin{exercise}[Do-almost-nothing kernelization: \itshape I can design and prove the correctness of kernelization algorithms by arguing that every yes- or every no-instance already satisfies the size constraint][\mandatory]
  Given an undirected graph $G = (V, E)$, a subset of vertices $X \subseteq V$ is a \emph{feedback vertex
  set} if removing~$X$ and all edges incident to~$X$ from~$G$ yields a forest. Show that the problem of deciding
  whether a graph has a feedback vertex set of size at most $k$ has a kernel with $O(k)$ vertices on regular,
  undirected graphs. (Note: The degree is not assumed to be constant!)
\end{exercise}

\begin{exercise}[High-degree reduction rules: \itshape I can design and prove the correctness of kernelization algorithms by using a suitable \enquote{high-degree} type reduction rule][\mandatory]
  Consider the following problem: Given a set of $n$ points in the plane, decide whether
  there is a set of $k$ lines such that every point lies on at least one line. Prove that this problem has a
  kernel of size $O(k^2)$.
\end{exercise}%

\textbf{Note:} Problems marked with \hard are especially challenging---they require that you invest a lot of time, play around with different ideas, and have a bit of luck that you find one that works. Best enjoyed with your favorite beverage and a friend. If you solve them, you can be proud of yourself. If you don't, you can still be proud of yourself for trying.

\begin{exercise}[Kernelization for Connected Vertex-Cover][\hard]
  Consider the problem from \ref{ex:CVC}.
  \begin{enumerate}
      \item Show that the problem admits a kernel with at most $2^k + O(k^2)$ vertices.
      \item\label{CVC-b} Show that if the input graph does not contain a $4$-cycle as a subgraph, then the problem admits a
      kernel with at most $O(k^2)$ vertices.
      \item Extend the argument from \ref{CVC-b} to show that, for every fixed $d \geq 2$, Connected Vertex-Cover restricted to graphs that do not contain the biclique $K_{d,d}$ admits a kernel with $O(k^d)$ vertices.
  \end{enumerate}
\end{exercise}

\newpage
\textbf{Note:} The following problems are meant to deepen your understanding of the course material. If you want, you can submit them to train your formal writing skills.

\begin{exercise}[Formal writing: \itshape I can write a formal proof when I am given a proof sketch]
  Let $G = (V, E)$ be a graph, and let $x \in \R^V$ be an
  optimum solution to the linear programming formulation of the vertex-cover problem, LPVC($G$). While $x$ is not necessarily half-integral, you will prove now that it can be turned into an optimal solution~$y$ that is half-integral. To this end, define
  $y\in\R^V$ as follows:
  \[
  y_v = \begin{cases}
  0 & \text{if } x_v < \frac{1}{2}\,, \\
  \frac{1}{2} & \text{if } x_v = \frac{1}{2}\,, \\
  1 & \text{if } x_v > \frac{1}{2}\,.
  \end{cases}
  \]
  Prove that $y$ is also an optimal solution to LPVC($G$).

  Hint: For $\delta>0$, define $\widetilde x\in\R^V$ with\[
    \widetilde x_v = \begin{cases}
    x_v-\delta & \text{if } x_v < \frac{1}{2}\,, \\
    \frac{1}{2} & \text{if } x_v = \frac{1}{2}\,, \\
    x_v+\delta & \text{if } x_v > \frac{1}{2}\,.
    \end{cases}
    \]
    Show for small enough $\delta$ that $\widetilde x$ is also an optimal feasible solution to LPVC($G$). Which setting of~$\delta$ leads~$\widetilde x$ to have fewer values that are not half-integral? Now argue inductively.
\end{exercise}


\begin{exercise}[Sunflower lemma: \itshape I can apply the sunflower lemma to design kernelization algorithms] %[\mandatory]
  First, read Section 2.6 about the Sunflower Lemma in the book, including its application to the $d$-hitting set problem.
  Then, using similar ideas, solve the following exercise on the $d$-set packing problem:
  You are given an integer $k$ and a family of subsets $\mathcal{A}$ of a universe $U$, where each set in $\mathcal{A}$ is of size at most $d$. Decide whether there exist $k$ sets $S_1,\ldots,S_k \in \mathcal{A}$ that are pairwise disjoint.
  Use the Sunflower Lemma to obtain a kernel for this problem with $f(d)\cdot k^d$ sets, for some computable $f$.
\end{exercise}

\end{document}

